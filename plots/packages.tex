\usepackage{tikz}
\usepackage{pgfplots}
\usepackage{gnuplottex}[2016/08/21]%must be before listings package; min date is for compat with tikzexternalize
\usepackage{pgfkeys}
\usepackage{pgfplotstable}
\pgfplotsset{compat=1.15}
\usepackage{currfile}
\usepackage{filecontents}

\usepgfplotslibrary{external}
\tikzexternalize
\tikzsetfigurename{\tikzexternalrealjob-\currfilebase-figure}

\usepackage{hyperref}
\usepackage{unicode-math}
\usepackage{amsmath}
\usepackage{newunicodechar}
\newcommand{\newunicodecharpdftex}[2]{\newunicodechar{#1}{\texorpdfstring{#2}{#1}}}
\newunicodecharpdftex{Α}{\ensuremath{\Alpha}}
\newunicodecharpdftex{Β}{\ensuremath{\Beta}}
\newunicodecharpdftex{Γ}{\ensuremath{\Gamma}}
\newunicodecharpdftex{Δ}{\ensuremath{\Delta}}
\newunicodecharpdftex{Ε}{\ensuremath{\Epsilon}}
\newunicodecharpdftex{Ζ}{\ensuremath{\Zeta}}
\newunicodecharpdftex{Η}{\ensuremath{\Eta}}
\newunicodecharpdftex{Θ}{\ensuremath{\Theta}}
\newunicodecharpdftex{Ι}{\ensuremath{\Iota}}
\newunicodecharpdftex{Κ}{\ensuremath{\Kappa}}
\newunicodecharpdftex{Λ}{\ensuremath{\Lambda}}
\newunicodecharpdftex{Μ}{\ensuremath{\Mu}}
\newunicodecharpdftex{Ν}{\ensuremath{\Nu}}
\newunicodecharpdftex{Ξ}{\ensuremath{\Xi}}
\newunicodecharpdftex{Ο}{\ensuremath{\Omicron}}
\newunicodecharpdftex{Π}{\ensuremath{\Pi}}
\newunicodecharpdftex{Ρ}{\ensuremath{\Rho}}
\newunicodecharpdftex{Σ}{\ensuremath{\Sigma}}
\newunicodecharpdftex{Τ}{\ensuremath{\Tau}}
\newunicodecharpdftex{Υ}{\ensuremath{\Upsilon}}
\newunicodecharpdftex{Φ}{\ensuremath{\Phi}}
\newunicodecharpdftex{Χ}{\ensuremath{\Chi}}
\newunicodecharpdftex{Ψ}{\ensuremath{\Psi}}
\newunicodecharpdftex{Ω}{\ensuremath{\Omega}}
\newunicodecharpdftex{α}{\ensuremath{\alpha}}
\newunicodecharpdftex{β}{\ensuremath{\beta}}
\newunicodecharpdftex{γ}{\ensuremath{\gamma}}
\newunicodecharpdftex{δ}{\ensuremath{\delta}}
\newunicodecharpdftex{ϵ}{\ensuremath{\epsilon}}
\newunicodecharpdftex{ε}{\ensuremath{\varepsilon}}
\newunicodecharpdftex{ζ}{\ensuremath{\zeta}}
\newunicodecharpdftex{η}{\ensuremath{\eta}}
\newunicodecharpdftex{θ}{\ensuremath{\theta}}
\newunicodecharpdftex{ϑ}{\ensuremath{\vartheta}}
\newunicodecharpdftex{ι}{\ensuremath{\iota}}
\newunicodecharpdftex{κ}{\ensuremath{\kappa}}
\newunicodecharpdftex{ϰ}{\ensuremath{\varkappa}}
\newunicodecharpdftex{λ}{\ensuremath{\lambda}}
\newunicodecharpdftex{μ}{\ensuremath{\mu}}
\newunicodecharpdftex{ν}{\ensuremath{\nu}}
\newunicodecharpdftex{ξ}{\ensuremath{\xi}}
\newunicodecharpdftex{ο}{\ensuremath{\omicron}}
\newunicodecharpdftex{π}{\ensuremath{\pi}}
\newunicodecharpdftex{ϖ}{\ensuremath{\varpi}}
\newunicodecharpdftex{ρ}{\ensuremath{\rho}}
\newunicodecharpdftex{ϱ}{\ensuremath{\varrho}}
\newunicodecharpdftex{σ}{\ensuremath{\sigma}}
\newunicodecharpdftex{ς}{\ensuremath{\varsigma}}
\newunicodecharpdftex{τ}{\ensuremath{\tau}}
\newunicodecharpdftex{υ}{\ensuremath{\upsilon}}
\newunicodecharpdftex{ϕ}{\ensuremath{\phi}}
\newunicodecharpdftex{φ}{\ensuremath{\varphi}}
\newunicodecharpdftex{χ}{\ensuremath{\chi}}
\newunicodecharpdftex{ψ}{\ensuremath{\psi}}
\newunicodecharpdftex{ω}{\ensuremath{\omega}}
\newunicodecharpdftex{∀}{\ensuremath{\forall}}
\newunicodecharpdftex{∃}{\ensuremath{\exists}}
\newunicodecharpdftex{→}{\ensuremath{\to}}
\newunicodecharpdftex{⇒}{\ensuremath{\Rightarrow}}
\newunicodecharpdftex{×}{\ensuremath{\times}}
\newunicodecharpdftex{∧}{\ensuremath{\wedge}}
\newunicodecharpdftex{⊢}{\ensuremath{\vdash}}
\newunicodecharpdftex{𝔹}{\ensuremath{\mathbb{B}}}
\newunicodecharpdftex{ℂ}{\ensuremath{\mathbb{C}}}
\newunicodecharpdftex{ℕ}{\ensuremath{\mathbb{N}}}
\newunicodecharpdftex{ℙ}{\ensuremath{\mathbb{P}}}
\newunicodecharpdftex{ℚ}{\ensuremath{\mathbb{Q}}}
\newunicodecharpdftex{ℝ}{\ensuremath{\mathbb{R}}}
\newunicodecharpdftex{ℤ}{\ensuremath{\mathbb{Z}}}
\newunicodecharpdftex{⁰}{\ensuremath{{}^0}}
\newunicodecharpdftex{¹}{\ensuremath{{}^1}}
\newunicodecharpdftex{²}{\ensuremath{{}^2}}
\newunicodecharpdftex{³}{\ensuremath{{}^3}}
\newunicodecharpdftex{⁴}{\ensuremath{{}^4}}
\newunicodecharpdftex{⁵}{\ensuremath{{}^5}}
\newunicodecharpdftex{⁶}{\ensuremath{{}^6}}
\newunicodecharpdftex{⁷}{\ensuremath{{}^7}}
\newunicodecharpdftex{⁸}{\ensuremath{{}^8}}
\newunicodecharpdftex{⁹}{\ensuremath{{}^9}}
\newunicodecharpdftex{₀}{\ensuremath{{}_0}}
\newunicodecharpdftex{₁}{\ensuremath{{}_1}}
\newunicodecharpdftex{₂}{\ensuremath{{}_2}}
\newunicodecharpdftex{₃}{\ensuremath{{}_3}}
\newunicodecharpdftex{₄}{\ensuremath{{}_4}}
\newunicodecharpdftex{₅}{\ensuremath{{}_5}}
\newunicodecharpdftex{₆}{\ensuremath{{}_6}}
\newunicodecharpdftex{₇}{\ensuremath{{}_7}}
\newunicodecharpdftex{₈}{\ensuremath{{}_8}}
\newunicodecharpdftex{₉}{\ensuremath{{}_9}}
\newunicodecharpdftex{⁺}{\ensuremath{{}^+}}
\newunicodecharpdftex{⁻}{\ensuremath{{}^-}}
\newunicodecharpdftex{⁼}{\ensuremath{{}^=}}
\newunicodecharpdftex{⁽}{\ensuremath{{}^(}}
\newunicodecharpdftex{⁾}{\ensuremath{{}^)}}
\newunicodecharpdftex{₊}{\ensuremath{{}_+}}
\newunicodecharpdftex{₋}{\ensuremath{{}_-}}
\newunicodecharpdftex{₌}{\ensuremath{{}_=}}
\newunicodecharpdftex{₍}{\ensuremath{{}_(}}
\newunicodecharpdftex{₎}{\ensuremath{{}_)}}

\newunicodecharpdftex{ᵃ}{\ensuremath{{}^{\text{a}}}}
\newunicodecharpdftex{ᵇ}{\ensuremath{{}^{\text{b}}}}
\newunicodecharpdftex{ᶜ}{\ensuremath{{}^{\text{c}}}}
\newunicodecharpdftex{ᵈ}{\ensuremath{{}^{\text{d}}}}
\newunicodecharpdftex{ᵉ}{\ensuremath{{}^{\text{e}}}}
\newunicodecharpdftex{ᶠ}{\ensuremath{{}^{\text{f}}}}
\newunicodecharpdftex{ᵍ}{\ensuremath{{}^{\text{g}}}}
\newunicodecharpdftex{ʰ}{\ensuremath{{}^{\text{h}}}}
\newunicodecharpdftex{ⁱ}{\ensuremath{{}^{\text{i}}}}
\newunicodecharpdftex{ʲ}{\ensuremath{{}^{\text{j}}}}
\newunicodecharpdftex{ᵏ}{\ensuremath{{}^{\text{k}}}}
\newunicodecharpdftex{ˡ}{\ensuremath{{}^{\text{l}}}}
\newunicodecharpdftex{ᵐ}{\ensuremath{{}^{\text{m}}}}
\newunicodecharpdftex{ⁿ}{\ensuremath{{}^{\text{n}}}}
\newunicodecharpdftex{ᵒ}{\ensuremath{{}^{\text{o}}}}
\newunicodecharpdftex{ᵖ}{\ensuremath{{}^{\text{p}}}}
\newunicodecharpdftex{ʳ}{\ensuremath{{}^{\text{r}}}}
\newunicodecharpdftex{ˢ}{\ensuremath{{}^{\text{s}}}}
\newunicodecharpdftex{ᵗ}{\ensuremath{{}^{\text{t}}}}
\newunicodecharpdftex{ᵘ}{\ensuremath{{}^{\text{u}}}}
\newunicodecharpdftex{ᵛ}{\ensuremath{{}^{\text{v}}}}
\newunicodecharpdftex{ʷ}{\ensuremath{{}^{\text{w}}}}
\newunicodecharpdftex{ˣ}{\ensuremath{{}^{\text{x}}}}
\newunicodecharpdftex{ʸ}{\ensuremath{{}^{\text{y}}}}
\newunicodecharpdftex{ᶻ}{\ensuremath{{}^{\text{z}}}}
\newunicodecharpdftex{ₐ}{\ensuremath{{}_{\text{a}}}}
\newunicodecharpdftex{ₑ}{\ensuremath{{}_{\text{e}}}}
\newunicodecharpdftex{ₕ}{\ensuremath{{}_{\text{h}}}}
\newunicodecharpdftex{ᵢ}{\ensuremath{{}_{\text{i}}}}
\newunicodecharpdftex{ⱼ}{\ensuremath{{}_{\text{j}}}}
\newunicodecharpdftex{ₖ}{\ensuremath{{}_{\text{k}}}}
\newunicodecharpdftex{ₗ}{\ensuremath{{}_{\text{l}}}}
\newunicodecharpdftex{ₘ}{\ensuremath{{}_{\text{m}}}}
\newunicodecharpdftex{ₙ}{\ensuremath{{}_{\text{n}}}}
\newunicodecharpdftex{ₒ}{\ensuremath{{}_{\text{o}}}}
\newunicodecharpdftex{ₚ}{\ensuremath{{}_{\text{p}}}}
\newunicodecharpdftex{ᵣ}{\ensuremath{{}_{\text{r}}}}
\newunicodecharpdftex{ₛ}{\ensuremath{{}_{\text{s}}}}
\newunicodecharpdftex{ₜ}{\ensuremath{{}_{\text{t}}}}
\newunicodecharpdftex{ᵤ}{\ensuremath{{}_{\text{u}}}}
\newunicodecharpdftex{ᵥ}{\ensuremath{{}_{\text{v}}}}
\newunicodecharpdftex{ₓ}{\ensuremath{{}_{\text{x}}}}

\newunicodecharpdftex{≅}{\ensuremath{\cong}}
\newunicodecharpdftex{∘}{\ensuremath{\circ}}


% to have the external picture be updated every time the input files are changed
\makeatletter
\newcommand{\einput}[1]{\@@input #1 \space}
\newcommand{\beginTikzpictureStamped}[2][]{%
    {%
        \everyeof{\noexpand}% IDK why \noexpand is the magic one, but I got it from http://mirrors.ibiblio.org/CTAN/macros/latex/contrib/oberdiek/catchfile.pdf
        \long\xdef\@tikzstamp{#2}%
    }%
    \def\@dobegintikzpicture{\begin{tikzpicture}[#1]}%
    \expandafter\@dobegintikzpicture\expandafter\def\expandafter\tikzstamp\expandafter{\@tikzstamp}%
}
\newcommand{\TikzpictureStamped}[3][]{\beginTikzpictureStamped[#1]{#2}#3\end{tikzpicture}}%
\makeatother


\usepackage{xstring} % for \IfStrEq, \IfInteger

%% exponential regression fit
\makeatletter
% https://tex.stackexchange.com/a/50113/2066
\newcommand*{\IsInteger}[3]{%
    \IfStrEq{#1}{ }{%
        #3% is a blank string
    }{%
    \IfInteger{#1}{#2}{#3}%
}%
}%
\newcommand{\pgftognucolumnset}[2]{%
    \IsInteger{\pgfkeysvalueof{#1}}{%
        % pgf 0-indexes columns, while gnuplot 1-indexes columns, so we add 1 to adjust
        \edef#2{\the\numexpr\pgfkeysvalueof{#1}+1\relax}%
    }{%
    \edef#2{(column("\pgfkeysvalueof{#1}"))}%
}%
}
\makeatletter
% \addplotexponentialregression[params for \addplot][default settings for a and b, also for x and y columns]{table file}
\NewDocumentCommand{\addplotexponentialregression}{ O{no markers} o m}{%
    \pgfkeyssetvalue{/addplotexponentialregression/x}{0}
    \pgfkeyssetvalue{/addplotexponentialregression/y}{1}
    \pgfkeyssetvalue{/addplotexponentialregression/a}{1}
    \pgfkeyssetvalue{/addplotexponentialregression/b}{1}
    \pgfkeys{/addplotexponentialregression/.cd,#2}
    \pgftognucolumnset{/addplotexponentialregression/x}{\@addplotexponentialregression@colx}%
    \pgftognucolumnset{/addplotexponentialregression/y}{\@addplotexponentialregression@coly}%
    \edef\@addplotexponentialregression@inita{\pgfkeysvalueof{/addplotexponentialregression/a}}%
    \edef\@addplotexponentialregression@initb{\pgfkeysvalueof{/addplotexponentialregression/b}}%
    \addplot [#1] gnuplot [raw gnuplot] { % allows arbitrary gnuplot commands
        f(x) = a*exp(b*x);     % Define the function to fit
        % Set reasonable starting values here
        a=\@addplotexponentialregression@inita;
        b=\@addplotexponentialregression@initb;
        fit f(x) '#3' u \@addplotexponentialregression@colx:\@addplotexponentialregression@coly\space via a,b; % Select the file
        stats '#3' u \@addplotexponentialregression@colx;
        plot [x=STATS_min:STATS_max] f(x);    % Specify the range to plot
        set print "#3-parameters.dat";  % Open a file to save the parameters
        print a, b;                  % Write the parameters to file
    };
    \pgfplotstableread{#3-parameters.dat}\parameters % Open the file Gnuplot wrote
    \pgfplotstablegetelem{0}{0}\of\parameters \xdef\pgfplotstableregressiona{\pgfplotsretval} % Get first element, save into \pgfplotstableregressiona
    \pgfplotstablegetelem{0}{1}\of\parameters \xdef\pgfplotstableregressionb{\pgfplotsretval}
}
% \addplotquadraticregression[params for \addplot][default settings for a and b and c, also for x and y columns]{table file}
\NewDocumentCommand{\addplotquadraticregression}{ O{no markers} o m}{%
    \pgfkeyssetvalue{/addplotquadraticregression/x}{0}
    \pgfkeyssetvalue{/addplotquadraticregression/y}{1}
    \pgfkeyssetvalue{/addplotquadraticregression/a}{1}
    \pgfkeyssetvalue{/addplotquadraticregression/b}{1}
    \pgfkeyssetvalue{/addplotquadraticregression/c}{1}
    \pgfkeys{/addplotquadraticregression/.cd,#2}
    \pgftognucolumnset{/addplotquadraticregression/x}{\@addplotquadraticregression@colx}%
    \pgftognucolumnset{/addplotquadraticregression/y}{\@addplotquadraticregression@coly}%
    \edef\@addplotquadraticregression@inita{\pgfkeysvalueof{/addplotquadraticregression/a}}%
    \edef\@addplotquadraticregression@initb{\pgfkeysvalueof{/addplotquadraticregression/b}}%
    \edef\@addplotquadraticregression@initc{\pgfkeysvalueof{/addplotquadraticregression/c}}%
    \addplot [#1] gnuplot [raw gnuplot] { % allows arbitrary gnuplot commands
        f(x) = a*x**2+b*x+c;     % Define the function to fit
        % Set reasonable starting values here
        a=\@addplotquadraticregression@inita;
        b=\@addplotquadraticregression@initb;
        c=\@addplotquadraticregression@initc;
        fit f(x) '#3' u \@addplotquadraticregression@colx:\@addplotquadraticregression@coly\space via a,b,c; % Select the file
        stats '#3' u \@addplotquadraticregression@colx;
        plot [x=STATS_min:STATS_max] f(x);    % Specify the range to plot
        set print "#3-parameters.dat";  % Open a file to save the parameters
        print a, b, c;                  % Write the parameters to file
    };
    \pgfplotstableread{#3-parameters.dat}\parameters % Open the file Gnuplot wrote
    \pgfplotstablegetelem{0}{0}\of\parameters \xdef\pgfplotstableregressiona{\pgfplotsretval} % Get first element, save into \pgfplotstableregressiona
    \pgfplotstablegetelem{0}{1}\of\parameters \xdef\pgfplotstableregressionb{\pgfplotsretval}
    \pgfplotstablegetelem{0}{2}\of\parameters \xdef\pgfplotstableregressionc{\pgfplotsretval}
}
% \addplotcubicregression[params for \addplot][default settings for a and b and c and d, also for x and y columns]{table file}
\NewDocumentCommand{\addplotcubicregression}{ O{no markers} o m}{%
    \pgfkeyssetvalue{/addplotcubicregression/x}{0}
    \pgfkeyssetvalue{/addplotcubicregression/y}{1}
    \pgfkeyssetvalue{/addplotcubicregression/a}{1}
    \pgfkeyssetvalue{/addplotcubicregression/b}{1}
    \pgfkeyssetvalue{/addplotcubicregression/c}{1}
    \pgfkeyssetvalue{/addplotcubicregression/d}{1}
    \pgfkeys{/addplotcubicregression/.cd,#2}
    \pgftognucolumnset{/addplotcubicregression/x}{\@addplotcubicregression@colx}%
    \pgftognucolumnset{/addplotcubicregression/y}{\@addplotcubicregression@coly}%
    \edef\@addplotcubicregression@inita{\pgfkeysvalueof{/addplotcubicregression/a}}%
    \edef\@addplotcubicregression@initb{\pgfkeysvalueof{/addplotcubicregression/b}}%
    \edef\@addplotcubicregression@initc{\pgfkeysvalueof{/addplotcubicregression/c}}%
    \edef\@addplotcubicregression@initd{\pgfkeysvalueof{/addplotcubicregression/d}}%
    \addplot [#1] gnuplot [raw gnuplot] { % allows arbitrary gnuplot commands
        f(x) = a*x**3+b*x**2+c*x+d;     % Define the function to fit
        % Set reasonable starting values here
        a=\@addplotcubicregression@inita;
        b=\@addplotcubicregression@initb;
        c=\@addplotcubicregression@initc;
        d=\@addplotcubicregression@initd;
        fit f(x) '#3' u \@addplotcubicregression@colx:\@addplotcubicregression@coly\space via a,b,c,d; % Select the file
        stats '#3' u \@addplotcubicregression@colx;
        plot [x=STATS_min:STATS_max] f(x);    % Specify the range to plot
        set print "#3-parameters.dat";  % Open a file to save the parameters
        print a, b, c, d;                  % Write the parameters to file
    };
    \pgfplotstableread{#3-parameters.dat}\parameters % Open the file Gnuplot wrote
    \pgfplotstablegetelem{0}{0}\of\parameters \xdef\pgfplotstableregressiona{\pgfplotsretval} % Get first element, save into \pgfplotstableregressiona
    \pgfplotstablegetelem{0}{1}\of\parameters \xdef\pgfplotstableregressionb{\pgfplotsretval}
    \pgfplotstablegetelem{0}{2}\of\parameters \xdef\pgfplotstableregressionc{\pgfplotsretval}
    \pgfplotstablegetelem{0}{3}\of\parameters \xdef\pgfplotstableregressiond{\pgfplotsretval}
}
% regress on y = a (x!)
% \addplotfactorialregression[params for \addplot][default settings for a, also for x and y columns]{table file}
\NewDocumentCommand{\addplotfactorialregression}{ O{no markers} o m}{%
    \pgfkeyssetvalue{/addplotquadraticregression/x}{0}
    \pgfkeyssetvalue{/addplotquadraticregression/y}{1}
    \pgfkeyssetvalue{/addplotquadraticregression/a}{1}
    \pgfkeys{/addplotquadraticregression/.cd,#2}
    \pgftognucolumnset{/addplotquadraticregression/x}{\@addplotquadraticregression@colx}%
    \pgftognucolumnset{/addplotquadraticregression/y}{\@addplotquadraticregression@coly}%
    \edef\@addplotquadraticregression@inita{\pgfkeysvalueof{/addplotquadraticregression/a}}%
    \addplot [#1] gnuplot [raw gnuplot] { % allows arbitrary gnuplot commands
        f(x) = a*gamma(x+1);     % Define the function to fit
        % Set reasonable starting values here
        a=\@addplotquadraticregression@inita;
        fit f(x) '#3' u \@addplotquadraticregression@colx:\@addplotquadraticregression@coly\space via a; % Select the file
        stats '#3' u \@addplotquadraticregression@colx;
        plot [x=STATS_min:STATS_max] f(x);    % Specify the range to plot
        set print "#3-parameters.dat";  % Open a file to save the parameters
        print a;                  % Write the parameters to file
    };
    \pgfplotstableread{#3-parameters.dat}\parameters % Open the file Gnuplot wrote
    \pgfplotstablegetelem{0}{0}\of\parameters \xdef\pgfplotstableregressiona{\pgfplotsretval} % Get first element, save into \pgfplotstableregressiona
}
\makeatother
